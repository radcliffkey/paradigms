%%% Titulní strana práce

\pagestyle{empty}
\begin{center}

\large

Charles University in Prague

\medskip

Faculty of Mathematics and Physics

\vfill

{\bf\Large MASTER'S THESIS}

\vfill

\centerline{\hspace{18mm}\mbox{\includegraphics[width=60mm]{logo.pdf}}}

\vfill
\vspace{5mm}

{\LARGE Radoslav Klíč}

\vspace{15mm}

% Název práce přesně podle zadání
{\LARGE\bfseries Acquisition of inflectional paradigms with minimal supervision}

\vfill

% Název katedry nebo ústavu, kde byla práce oficiálně zadána
% (dle Organizační struktury MFF UK)
Institute of Formal and Applied Linguistics

\vfill

\begin{tabular}{rl}

Supervisor of the master's thesis: & RNDr. Jiří Hana Ph.D.\\
\noalign{\vspace{2mm}}
Study programme: & Computer Science\\
\noalign{\vspace{2mm}}
Specialisation: & Computational and Formal Linguistics\\
\end{tabular}

\vfill

% Zde doplňte rok
Prague 2012

\end{center}

\newpage

%%% Následuje vevázaný list -- kopie podepsaného "Zadání diplomové práce".
%%% Toto zadání NENÍ součástí elektronické verze práce, nescanovat.

%%% Na tomto místě mohou být napsána případná poděkování (vedoucímu práce,
%%% konzultantovi, tomu, kdo zapůjčil software, literaturu apod.)

\openright
\onehalfspacing
\noindent
I would like to thank my supervisor, doctor Hana, for professional guidance, a lot of useful advice and setting an example of a professional scientist to me.

\newpage

%%% Strana s čestným prohlášením k diplomové práci

\vglue 0pt plus 1fill

\noindent
I declare that I carried out this master's thesis independently, and only with the cited sources, literature and other professional sources.

\medskip\noindent
I understand that my work relates to the rights and obligations under the Act No. 121/2000 Coll., the Copyright Act, as amended, in particular the fact that the Charles University in Prague has the right to conclude a license agreement on the use of this work as a school work pursuant to Section 60 paragraph 1 of the Copyright Act.

\vspace{10mm}

\hbox{\hbox to 0.5\hsize{%
In ........ date ............
\hss}\hbox to 0.5\hsize{%
signature
\hss}}

\vspace{20mm}
\newpage

%%% Povinná informační strana diplomové práce
\singlespacing
\vbox to 0.5\vsize{
\setlength\parindent{0mm}
\setlength\parskip{5mm}

Název práce:
Automatické osvojení vzorů s minimální supervizí
% přesně dle zadání

Autor: Radoslav Klíč

Katedra: Ústav formální a aplikované lingvistiky (ÚFAL)


Vedoucí diplomové práce:
RNDr. Jiří Hana Ph.D., ÚFAL
% dle Organizační struktury MFF UK, případně plný název pracoviště mimo MFF UK

\onehalfspacing
Abstrakt:
Diplomová práce popisuje algoritmus pro automatické osvojení vzorů s minimální supervizí, který vznikl rozšířením systému Paramor \citep{monson09}, fungujícího zcela bez supervize. Systém je modifikován, aby přijímal snadno dostupná data ve formě ohýbaných slov s označenou hranicí morfémů jako dodatečný vstup. Součástí práce je také knihovna pro hierarchické shlukování, která umožňuje kombinaci různých zdrojů informací. Přístup byl testován na češtině, slovin\-šti\-ně, němčině a katalánštině a vykázal zvýšenou F-míru v porovnáni se základním Paramorem.

Klíčová slova:
strojové učení, morfologie, fonologie, vzory ohýbaní slov
\onehalfspacing
\vss}\nobreak\vbox to 0.49\vsize{
\setlength\parindent{0mm}
\setlength\parskip{5mm}

Title:
Acquisition of inflectional paradigms with minimal supervision

Author:
Radoslav Klíč

Department:
Institute of Formal and Applied Linguistics
% dle Organizační struktury MFF UK v angličtině

Supervisor:
RNDr. Jiří Hana Ph.D., Institute of Formal and Applied Linguistics

\onehalfspacing
Abstract:
The thesis presents a semi-supervised morphology learner developed by extending Paramor \citep{monson09}, an unsupervised system, to accept easy to obtain manually provided data in the form of inflections with marked morpheme boundary. In addition, a hierarchical clustering framework allowing combination of multiple sources of information was developed as a part of the thesis. The approach was tested on Czech, Slovene, German and Catalan and has shown increased F-measure in comparison with the Paramor baseline.

Keywords:
machine learning, morphology, phonology, inflectional paradigms

\vss}

\newpage