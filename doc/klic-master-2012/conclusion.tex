\chapter{Conclusion}\label{chapter:conclusion}
The thesis presented an extension of a system for unsupervised morphology acquisition and a word clustering framework able to combine the system's output with modified edit distance. Modifications developed in the thesis enable the system to:
\begin{enumerate}
\item accept manually provided seed data in the form of inflections with marked morpheme boundary.
\item handle stem allomorphy using rules induced from the seed.
\end{enumerate} 

\noindent Testing on 4 languages from 3 language families showed that providing a small number of inflections leads to improvement in the performance of the system. The testing also uncovered some of the shortcomings of the approach. There is a number of issues future work can address:
\begin{itemize}
\item Currently, the system does not capture stem-internal changes such as German umlaut (\e{\e{Mutter/Mütter}}). Rules for such changes could be induced from the seed, with possibility to limit the characters from which and to which the changes may take place.
\item Frequencies of the words in the corpus are ignored, as well as the context they occur in.
Intelligent incorporation of statistical, contextual and semantic features of words or morphemes may lead to significantly higher quality of the analysis.
\end{itemize}
